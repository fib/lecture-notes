\documentclass{article}
\usepackage{amsmath}
<<<<<<< HEAD
\usepackage{amsthm}

\theoremstyle{definition}
\newtheorem{definition}{Definition}[section]

\title{CS220: Lecture Notes}
\author{Josh Kotler}
\date{last updated: \today}

\begin{document}
\maketitle
\section{Boolean Algebra}
\begin{itemize}
    \item Boolean algebra defines operations and rules for working with the set $\{0,1\}$.
\end{itemize}
\subsection{Boolean Operations and Functions}
\begin{description}
    \item[Complement] Denoted by a bar:
        \begin{align*}
            \overline{0} &= 1   &   \overline{1} &= 0
        \end{align*}
    \item[Boolean sum] Denoted as $+$ / $\text{OR}$:
        \begin{align*}
            1 + 1 &= 1   &   1 + 0 &= 1   \\
            0 + 1 &= 1   &   0 + 0 &= 0
        \end{align*}
    \item[Boolean product] Denoted as $\cdot$ / $\text{AND}$:
        \begin{align*}
            1 \cdot 1 &= 1   &   1 \cdot 0 &= 0   \\
            0 \cdot 1 &= 0   &   0 \cdot 0 &= 0
        \end{align*}
\end{description}
\begin{definition}[Boolean variable] Variable $x$ is a \textbf{Boolean variable} only
    if $x \in \{0,1\}$.
\end{definition}
\subsection{Identities}
\subsection{Defintition of a Boolean Algebra}
\begin{itemize}
    \item All the properties of Boolean functions and expression
        apply to other mathematical structures such as propositions
        and sets and the operations defined on them.
    \item If we can show that a particular structure is a Boolean algebra,
        then we know that all results established about Boolean algebras apply
        to this structure.
    \item For this purpose, we need an abstract definition of a Boolean algebra.
\end{itemize}
\begin{definition}[Boolean Algebra] A Boolean algebra is a set B with two binary operators
    $\land$ and $\lor$, elements 0 and 1, and a unary operation --
    such that the following properties hold fo all $x$, $y$, and $z$ in
    B:
    \begin{itemize}
        \item $x \lor 0 = x$ and $x \land 1 = x$ (identity laws).
    \end{itemize}
\end{definition}
=======

\title{CS220: Lecture 11 Notes}
\author{Josh Kotler}
\date{\today}

\begin{document}
    \maketitle
    \section{Boolean Operations and Functions}
    \begin{description}
        \item[Complement] Denoted by a bar.
            \begin{align*}
                \overline{0} &= 1   &   \overline{1} &= 0
            \end{align*}
        \item[Boolean sum] Denoted as $+$ / $\text{OR}$
            \begin{align*}
                1 + 1 &= 1   &   1 + 0 &= 1   \\
                0 + 1 &= 1   &   0 + 0 &= 0
            \end{align*}
        \item[Boolean product] Denoted as $\cdot$ / $\text{AND}$
            \begin{align*}
                1 \cdot 1 &= 1   &   1 \cdot 0 &= 0   \\
                0 \cdot 1 &= 0   &   0 \cdot 0 &= 0
            \end{align*}
    \end{description}
    \section{Identities}

>>>>>>> main
\end{document}
